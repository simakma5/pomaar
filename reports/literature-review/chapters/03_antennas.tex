\chapter{Front-end technologies}
\label{chap:frontend}

The antenna element serves as the fundamental physical interface of the radar sensor, defining the initial boundary conditions for signal fidelity. In the context of polarimetric MIMO automotive radar operating in the $\qtyrange{77}{81}{GHz}$ band, the antenna design is governed by a stringent conflict between electromagnetic purity and mass-producibility. While the synthesis of large virtual apertures relies on the placement of these elements, the quality of the polarimetric information is determined by the individual element's ability to maintain orthogonal polarization states over a wide field of view (FOV).

Achieving high cross-polarization discrimination (XPD) at millimetre-wave frequencies is notoriously difficult, particularly when constrained by the low-profile, cost-sensitive packaging requirements of the automotive industry. The antenna must not only exhibit robust isolation and gain stability but also mitigate the effects of surface waves and mutual coupling, which can degrade the orthogonality of the MIMO channels. Consequently, the choice of antenna technology is not merely a component selection but a system-level architectural decision that dictates the achievable dynamic range of the polarimetric radar.

\section{Design requirements and constraints}

The transition from standard automotive radar to fully polarimetric imaging imposes a specific set of performance metrics that narrow the field of viable antenna topologies. The primary driver is polarization purity across the scan volume: whereas legacy systems prioritize gain, a polarimetric sensor requires an XPD exceeding $\qty{20}{dB}$ across a wide azimuth FOV (typically up to $\pm\ang{60}$)~\parencite{tintiFullyPolarimetricAutomotive2024}. This is challenging because the geometric projection of polarization vectors' orthogonality naturally degrades at oblique angles~\parencite{zhaoOverviewPolarimetryApplication2025}.

Furthermore, these electromagnetic requirements must be reconciled with the realities of automotive integration. The antenna must be compatible with standard multi-layer printed circuit board (PCB) or package-level -- typically ball grid array (BGA) or embedded wafer lever BGA (eWLB) -- manufacturing processes, usually precluding bulky metallic waveguide flanges or machined horns. Furthermore, mutual coupling becomes a critical parameter in dense MIMO arrays; insufficient isolation between co-located orthogonal ports of dual-polarized antenna elements can lead to signal leakage that is mathematically indistinguishable from target depolarization. Finally, thermal stability is non-negotiable; the phase centre and resonant frequency must remain stable under the harsh temperature cycling of an automotive environment to prevent calibration drift in the virtual array manifold~\parencite{harterErrorAnalysisSelfcalibration2015}.

\section{Overview of radar front-end technology}

The literature presents a diverse array of structural architectures proposed for $\qty{79}{GHz}$ operation. These can be broadly categorized into planar printed structures -- which dominate the current commercial landscape -- and substrate-integrated or waveguide-based solutions that offer performance enhancements at the cost of manufacturing complexity.

\subsection{Planar printed technology}

The \emph{microstrip patch antenna} remains the standard radiating element for commercial automotive radar due to its seamless integration with low-cost multi-layer PCB processes~\parencite{waldschmidtAutomotiveRadarFirst2021}. In polarimetric applications, dual-polarized operation is typically achieved using orthogonal feeds (aperture-coupled or probe-fed) or proximity-coupled patches~\parencite{vallecchiDesignDualpolarizedSeriesfed2005,acimovicDualPolarizedMicrostripPatch2008,dengSeriesFedDualPolarizedSingleLayer2019}. While highly manufacturable, patch antennas suffer from intrinsic limitations at microwave frequencies: they are prone to high dielectric loss and surface wave excitation which degrade efficiency and coupling, and their narrow impedance bandwidth can be a bottleneck for wideband chirps~\parencite{zang77GHzFully2024}. Moreover, maintaining high XPD at wide angles is difficult due to the inevitable cross-polar radiation from the feed lines. Microstrip stub antennas, which utilize shaped stubs for improved impedance bandwidth and polarization control, pose a compelling alternative to conventional series patch arrays for compact, low-profile implementations~\parencite{tinti45degLinearlyPolarized2023}.

Antenna elements based on \emph{slotted radiation}, including cavity-backed slots and slanted slot arrays, offer an alternative planar approach. Slots naturally exhibit high polarization purity and can be more robust against mutual coupling than patches~\parencite{sunWidebandCavitySlottedWaveguide2024,liDesignImplementationSIW2019}. However, they are intrinsically single-polarized, and hence, integrating dual-polarized slot arrays often requires multi-layer feed networks~\parencite{ferreiraCompactDualCircularlyPolarized2017} that increase board complexity, and their bidirectional radiation pattern usually necessitates a back-cavity or reflector, adding to the vertical profile.

Emerging in the recent years, \emph{thin-film antennas} leverage advanced deposition techniques to realize ultra-thin radiating structures directly on the substrate~\parencite{khanHybridThinFilm2017}. While thin-film designs can support dual-polarization and flexible integration, they are typically very lossy at millimetre-wave frequencies, which limits their efficiency and practical deployment in automotive radar systems.

\subsection{Integrated waveguide technology}

To overcome the loss mechanisms of microstrip lines, \emph{substrate integrated waveguide} (SIW) and \emph{low temperature co-fired ceramic} (LTCC) technologies confine the electromagnetic field within a synthesized waveguide structure embedded in the dielectric. SIW antennas exhibit significantly lower transmission losses and superior element isolation compared to microstrip~\parencite{saeidi-maneshLowCrossPolarizationHighIsolation2019}. Their enclosed nature provides excellent shielding, making them robust against interference and temperature-induced deformations. However, the requisite via-fencing consumes considerable board real estate, posing a challenge implementing dense MIMO lattices without sacrificing grating lobe performance.

\emph{Gap waveguide} (GW) and \emph{ridge gap waveguide} (RGW) technologies represent a significant evolution in low-loss millimetre-wave design, addressing the assembly bottlenecks of traditional hollow waveguides. By utilizing an electromagnetic band-gap (EBG) surface, such as the \enquote{bed of nails} reported by~\textcite{kildalDesignExperimentalVerification2011}, to suppress parallel-plate modes, GW technology achieves the low loss of air-filled waveguides while eliminating the requirement for conductive electrical contact between waveguide layers.

This \enquote{contactless} characteristic creates a unique advantage for automotive radar: it relaxes the stringent mechanical flatness and assembly torque requirements that drive up the cost of traditional split-block waveguides. While early iterations relied on costly CNC milling, recent advancements in metallized plastic injection moulding and PCB-based implementations (using via fences or mushroom structures) have drastically reduced fabrication costs, as validated in the industrial study by~\parencite{bencivenniHighVolumeManufacturingMetallized2023}. Consequently, GWs are transitioning from academic demonstrators to commercial viability, offering a robust solution for high-efficiency, fully metallic antenna arrays capable of withstanding the harsh thermal and vibrational environments of automotive sensing~\parencite{yongOverviewRecentDevelopment2023}.

This \enquote{contactless} characteristic creates a unique advantage for automotive radar: it relaxes the stringent mechanical flatness and assembly torque requirements that drive up the cost of traditional split-block waveguides. While early iterations relied on costly CNC milling, recent advancements have enabled a transition to metallized plastic injection moulding. As detailed in industrial studies by~\textcite{huegel3DWaveguideMetallized2022,bencivenniHighVolumeManufacturingMetallized2023}, this approach allows for the rapid prototyping of complex layers via 3D printing before scaling to high-precision injection moulding for mass production. This workflow effectively shifts the complexity from assembly to the mould-design phase, resulting in high-efficiency, air-filled waveguide arrays commercially viable for the harsh thermal and vibrational environments of automotive sensing~\parencite{yongOverviewRecentDevelopment2023}.

\subsection{Volumetric antennas}

While impractical for conformal automotive integration, \emph{horn antennas} and \emph{dielectric resonator antennas} (DRAs) serve as important benchmarks, hence their use in polarimetric system demonstrators, such as those presented by~\textcite{tintiFullyPolarimetricAutomotive2024,trummerPerformanceAnalysis792017}. Horns exhibit outstanding gain and polarization discrimination but are volumetrically incompatible with bumper-integrated sensors. DRAs offer wide bandwidth and high radiation efficiency by eliminating metallic losses~\parencite{kishkWidebandTruncatedTetrahedron2003}, yet they face challenges regarding mechanical robustness and the precision assembly required to mount 3D dielectric blocks onto a planar radar transceiver.

\paragraph{Launcher-in-package technology.} In recent years, launcher-in-package (LiP) technology has emerged as a promising solution to the traditional challenges of integrating waveguide antennas with monolithic microwave integrated circuits (MMICs). By embedding the waveguide transition directly within the package substrate, LiP minimizes the number of RF transitions, thereby reducing insertion loss and reflection typically associated with conventional chip-to-waveguide interfaces~\parencite{mohammedAdvancementsMmWaveTechnology2024}. While still an emerging technology, this advancement enables more efficient coupling between the transceiver and the antenna, making waveguide-based solutions more viable for compact automotive radar systems.

\section{Comparative analysis of antenna technologies}

A critical review of the literature reveals that antenna performance is not determined by a single design choice, but rather by the interaction between two distinct layers: the \emph{radiating element} and the \emph{integration platform}. In many reported designs, these two aspects are conflated. To provide a clearer map of the design space, this analysis decouples these layers, evaluating them separately before synthesizing the state-of-the-art findings.

\subsection{Radiating element typologies}
The choice of radiating element dictates the intrinsic bandwidth, polarization discrimination, and radiation pattern of the sensor. \Cref{tab:radiating_elements} categorizes the fundamental element types found in automotive radar literature.

While microstrip patches dominate commercial implementations due to their low profile, they are intrinsically narrowband and prone to surface wave excitation. In contrast, volumetric radiators like horns and DRAs offer superior bandwidth and polarization stability but face severe integration penalties. Slot radiators occupy a middle ground, offering high XPD but requiring complex back-cavity structures to suppress bidirectional radiation.

\subsection{Integration platforms}
The integration platform defines the feeding structure of the antenna system: it determines insertion loss, isolation between MIMO channels, and thermal stability. As shown in \cref{tab:integration_platforms}, the industry standard (microstrip/PCB) trades performance for cost, while academic research heavily favours waveguide-based structures to maximize signal fidelity.

\begin{table}[ht]
    \centering
    \caption{Comparison of radiating element types for polarimetric radar}
    \label{tab:radiating_elements}
    \small
    \begin{tabular}{p{3.8cm} p{1.8cm} p{1.8cm} p{1.8cm} >{\footnotesize\raggedright\arraybackslash}p{3cm}}
        \toprule
        \textbf{Element type} & \textbf{BW (\%)} & \textbf{XPD (dB)} & \textbf{Profile} & {\small\textbf{References}} \\
        \midrule
        Edge-fed patch & $<5$ & 15--20 & Very low & \parencite{dashDesignSeriesFed2021,baur77GHzAutomotive2013,hambergerMixedCircularLinear2019} \\
        \addlinespace
        Aperture-coupled patch & 5--15 & 20--30 & Low & \parencite{acimovicDualPolarizedMicrostripPatch2008,puskely5GSIWBasedPhased2022,animHighGainMillimeterWavePatch2021,karamiDevelopingBroadbandMicrostrip2022a,saeidi-maneshLowCrossPolarizationHighIsolation2019} \\
        \addlinespace
        Microstrip stub & $<5$ & 20--25 & Very low & \parencite{tinti45degLinearlyPolarized2023,salucciSplineShapedMicrostripEdgeFed2024} \\
        \addlinespace
        Thin-film antenna & $\sim10$ & 20--30 & Very low & \parencite{khanHybridThinFilm2017} \\
        \addlinespace
        Slotted waveguide & $<5$ & 15--25 & Low & \parencite{hehenberger77GHzFMCWMIMO2020,bauer79GHzResonantLaminated2013} \\
        \addlinespace
        Cavity-backed slot & $<5$ & $>25$ & Moderate & \parencite{ferreiraCompactDualCircularlyPolarized2017,sunWidebandCavitySlottedWaveguide2024,mosalanejadMultiLayerPCBBowTie2020,liDesignImplementationSIW2019,yangMillimeterWaveDualPolarizedDifferentially2020} \\
        \addlinespace
        Horn waveguide & $>15$ & $>25$ & High (3D) & \parencite{trummerPerformanceAnalysis792017,tintiFullyPolarimetricAutomotive2024,zhaoSIWQuasiPyramidHorn2024} \\
        \addlinespace
        Dielectric resonator & 10--15 & $\sim20$ & High (3D) & \parencite{guoDualpolarizedDielectricResonator2003,kishkWidebandTruncatedTetrahedron2003} \\
        \bottomrule
    \end{tabular}
\end{table}

\begin{table}[ht]
    \centering
    \caption{Comparison of integration and feeding platforms}
    \label{tab:integration_platforms}
    \small
    \begin{tabular}{p{3.4cm} p{2cm} p{2cm} p{2cm} >{\footnotesize\raggedright\arraybackslash}p{3.2cm}}
        \toprule
        \textbf{Platform} & \textbf{Loss} & \textbf{Isolation} & \textbf{Complexity} & {\small\textbf{References}} \\
        \midrule
        Planar PCB & High & Low & Very low & \parencite{hambergerMixedCircularLinear2019,dashDesignSeriesFed2021,baur77GHzAutomotive2013,tinti45degLinearlyPolarized2023,salucciSplineShapedMicrostripEdgeFed2024} \\
        \addlinespace
        Multi-layer PCB & Moderate & Moderate & Moderate & \parencite{aliakbari79GHzMultilayer2022,mosalanejadMultiLayerPCBBowTie2020,saeidi-maneshLowCrossPolarizationHighIsolation2019} \\
        \addlinespace
        Metallic waveguide & Very low & Very high & Moderate & \parencite{tintiFullyPolarimetricAutomotive2024,trummerPerformanceAnalysis792017,schulwitzMillimeterWaveDualPolarized2006,sunWidebandCavitySlottedWaveguide2024} \\
        \addlinespace
        SIW & Moderate & High & Moderate & \parencite{puskely5GSIWBasedPhased2022,karamiDevelopingBroadbandMicrostrip2022a,hehenberger77GHzFMCWMIMO2020,zhaoSIWQuasiPyramidHorn2024,liDesignImplementationSIW2019,yangMillimeterWaveDualPolarizedDifferentially2020} \\
        \addlinespace
        LTCC & Low & High & High & \parencite{bauer79GHzResonantLaminated2013} \\
        \addlinespace
        Gap waveguide & Low & Very high & High & \parencite{renAutomotivePolarimetricRadar2024,zang77GHzFully2024} \\
        \bottomrule
    \end{tabular}
\end{table}

\section{Synthesis of state-of-the-art trends}

Analysing the intersection of these element and platform choices reveals a clear dichotomy in the current state of the art.

\paragraph{Industrial baseline.} Despite their electromagnetic limitations, microstrip patch arrays remain the incumbent solution for mass-market automotive radar. The manufacturing maturity of standard PCB processes allows for the low-cost integration of complex series-fed or series-parallel networks directly alongside the MMIC~\parencite{hartnerReliabilityPerformanceWafer2019}. However, for polarimetric applications, this convenience comes at a cost: standard patches struggle to maintain XPD beyond $\pm\qty{30}{\degree}$ scan angles~\parencite{yangMillimeterWaveDualPolarizedDifferentially2020}, and the mutual coupling between closely spaced elements in a dense MIMO array introduces phase errors that degrade virtual aperture synthesis and the performance of signal processing algorithms, as explored by~\textcite{arnoldEffectAntennaMutual2019}.

\paragraph{Performance frontier.} Recent academic literature identifies GW and RGW as the primary candidates for overcoming the inherent dielectric losses and surface-wave coupling of high-frequency PCB technology. By creating a prohibiting EBG, GW structures suppress parallel-plate modes entirely without requiring a galvanic connection between the waveguide layers. This \enquote{contactless} property is a critical industrial enabler; it significantly relaxes the mechanical assembly tolerances that make traditional hollow waveguides cost-prohibitive for high-volume production. Through the adoption of metallized injection moulding and high-precision plastic stamping, the manufacturing complexity is shifted from individual assembly to the mould-design phase, allowing for the low-cost mass production of air-filled structures. Consequently, GW technology, particularly in combination with the LiP technology, represents the emerging \enquote{gold standard} for next-generation MIMO arrays, providing the polarimetric fidelity and thermal robustness required for high-resolution automotive sensing within a commercially viable form factor~\parencite{renGapwaveguideAutomotiveImaging2023}.

\paragraph{The compromise.} SIW architectures effectively bridge the gap between planar and volumetric designs. By synthesizing a dielectric-filled waveguide using periodic via fences, SIW provides the electromagnetic shielding and low-loss characteristics of metallic waveguides while retaining the manufacturability of standard PCBs. The versatility of this platform is exemplified by~\textcite{zhaoSIWQuasiPyramidHorn2024}, who demonstrated that multi-layer PCB stacks can be used to synthesize quasi-pyramidal horn antennas entirely within the substrate. 

Furthermore, the layered structure of SIW facilitates advanced hybrid topologies. For instance, designs by~\textcite{puskely5GSIWBasedPhased2022,yangMillimeterWaveDualPolarizedDifferentially2020} combine the isolation benefits of an integrated waveguide feed with the beam-shaping flexibility of printed elements, using coupling slots in a sandwiched ground layer to excite parasitic patches on the surface.

In parallel, LTCC technology offers a similar volumetric approach but with superior thermal stability and dimensional tolerance compared to organic substrates, making it attractive for highly integrated package-level antennas, albeit at a higher implementation cost.

\section{Emerging paradigms}
\label{sec:emerging_paradigms}

Beyond the established categories of printed and waveguide-based elements, several exploratory technologies are expanding the design space for automotive radar. These paradigms prioritize the synthesis of ideal radiation characteristics over traditional fabrication constraints.

\paragraph{Magneto-electric dipoles (MED).} Originally developed to overcome the narrow bandwidth of standard patch antennas, the magneto-electric dipole has recently been adapted for millimetre-wave radar to address the specific requirements of polarimetric fidelity. By combining a planar electric dipole with a complementary magnetic dipole (typically synthesized via shorted patches or via-fenced slots), MEDs function effectively as Huygens sources.

This complementary excitation yields two distinct advantages for automotive sensing. First, it achieves wide impedance bandwidths (usually $>\qty{20}{\percent}$), easily covering the full $\qtyrange{76}{81}{GHz}$ band required for high-resolution 4D imaging~\parencite{li60GHzDualPolarizedTwoDimensional2016}. Second, and more critically for polarimetry, MEDs exhibit nearly identical E- and H-plane radiation patterns with low cross-polarization. This pattern symmetry ensures that the radar's \enquote{view} of a target is consistent regardless of polarization orientation, potentially simplifying the calibration matrices required for precise direction-of-arrival (DOA) estimation.

\paragraph{Metasurface-enhanced isolation.} As MIMO arrays become denser to support higher angular resolution, mutual coupling via surface waves becomes a dominant source of error. Designers are increasingly leveraging metasurfaces -- periodic sub-wavelength structures etched into the PCB ground or top layer -- to manipulate these boundary conditions.

Rather than functioning as radiating elements themselves, these structures often act as EBG filters or soft surfaces. By presenting a high surface impedance to grazing waves, they effectively trap or dissipate surface currents between adjacent patch elements. Recent implementations have demonstrated isolation improvements of $\qty{10}{dB}$ to $\qty{20}{dB}$ with negligible impact on the primary radiation pattern, offering a planar solution to the coupling problems that traditionally required metallic wall isolation~\parencite{mohamedPerfectIsolationPerformance2019}.

\paragraph{Additive manufacturing and integrated lenses.} Recent advances in high-precision additive manufacturing are transforming the landscape of millimetre-wave antenna prototyping. Unlike traditional machining or moulding, 3D printing enables the fabrication of complex waveguide structures with intricate internal geometries, such as non-standard transitions, that are otherwise impossible to manufacture. While currently more suited to rapid prototyping than high-volume automotive deployment, this technology offers a unique pathway for validating complex volumetric designs before committing to expensive tooling.

A compelling application of this paradigm is the direct 3D printing of integrated lens antennas, where dielectric lenses, such as Luneburg or Maxwell fish-eye profiles, are fabricated as part of the antenna structure~\parencite{choSeriesFedIntegratedLens2024}. This approach enables volumetric gain enhancement within a compact footprint, but remains limited by the loss tangent and surface roughness of available printable materials, which can introduce additional losses and phase errors at $\qty{79}{GHz}$. While promising for prototyping and specialized applications, further advances in printable materials and post-processing are needed for widespread automotive deployment.

\section{Design implications}
The survey of the literature leads to a specific set of design directions for the polarimetric MIMO demonstrator targeted in this work. While GW technology offers the highest theoretical performance, its fabrication complexity poses a risk for rapid prototyping. Conversely, microstrip patches offer a low-risk, low-cost baseline but are likely to bottleneck the polarimetric dynamic range of the system.

Therefore, a pragmatic high-performance approach points toward SIW-based topologies or advanced cavity-backed slot designs. These architectures offer the requisite isolation and polarization stability to validate sparse polarimetric MIMO algorithms, while remaining within the bounds of standard PCB fabrication capabilities.

The preceding review captures a vast and rapidly evolving design space for $\qty{79}{GHz}$ automotive radar. It is evident that emerging technologies, particularly GW as an integration platform and MEDs as radiating elements, are maturing into robust solutions that address the traditional losses and bandwidth limitations of millimetre-wave circuitry. These technologies represent the state of the art of electromagnetic design, offering superior isolation and polarization purity that are highly attractive for next-generation sensing.

However, the architectural choices for the specific research encompassed by this project are governed by a different set of optimization criteria. The primary objective of this doctoral work is the extraction of dynamic, Doppler-resolved polarimetric signatures using a large-aperture MIMO radar. In this context, the antenna array is not the primary subject of innovation but rather an enabler -- a critical functional block that must guarantee reliable data acquisition to validate the signal processing techniques of polarimetric decomposition and signature extraction.

Consequently, the focus is placed on the ability to rapidly prototype a complete, reliable polarimetric system with imaging capabilities, rather than employing experimental radiating elements that may introduce manufacturing uncertainties or integration delays. To align with this objective, the following design decisions have been established:

\begin{itemize}
    \item \textbf{Adoption of planar technology:} To attain industrial support in terms of manufacturability and compatibility with standard calibration routines, the design is restricted to planar technologies (PCB/LTCC). While acknowledging the superior efficiency of air-filled waveguides, the mature fabrication processes of printed circuits ensure that the large-scale MIMO array can be produced with consistent tolerances, which is a prerequisite for accurate array manifold calibration.

    \item \textbf{Selection of radiating mechanism:} Within the planar domain, the design strategy envisions a multi-layer SIW stack-up as the high-performance contender. Specifically, slotted radiation serves as the main mechanism, potentially augmented with parasitic patches on the top layer to shape the individual beam pattern and improve impedance bandwidth, a technique supported by recent studies on hybrid SIW-patch topologies, such as works of~\textcite{puskely5GSIWBasedPhased2022,yangMillimeterWaveDualPolarizedDifferentially2020}. This approach strikes a balance between the isolation benefits of substrate-integrated waveguides and the low profile of printed elements.

    \item \textbf{Iterative development strategy:} To mitigate risk, particularly in the early phases of the project, the front-end design will leverage the established experience of dealing with uniform patch array cascades. This approach mirrors the proven architectures employed by industry leaders, such as Texas Instruments~\parencite{texasInstrumentsAwrxCascadedRadar2020} and Continental, for imaging automotive radar. By starting with a known baseline -- standard series-fed microstrip patches -- the research ensures a stable platform for initial data collection and algorithm development, allowing the focus to remain on the signal processing challenges of high-fidelity polarimetric measurements resolved in Doppler.
\end{itemize}

In summary, while the literature points toward GW and MED as the future of automotive electromagnetics, this research purposefully selects established planar implementations to prioritize system-level reliability and the integrity of the polarimetric signal processing pipeline.