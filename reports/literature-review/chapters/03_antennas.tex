\chapter{Antenna types for polarimetric MIMO automotive radar}
\label{chap:antennas}
Automotive radar systems operating in the 77--81~GHz band demand compact, highly integrated antennas that offer robust dual-polarization capabilities for polarimetric sensing. Achieving high cross-polarization discrimination (XPD), wide field-of-view (FoV), and manufacturability within stringent size and cost constraints presents significant design challenges. The radar antennas must balance beam shaping, mutual coupling suppression, and thermal/mechanical stability to enable accurate object classification in advanced driver assistance systems (ADAS). This review critically evaluates existing antenna technologies recognized for their suitability and limitations in this application domain.

% ============================================================================
\section{Scope and requirements}
\label{sec:ant_scope}
\todo{Blend this in with the introduction\ldots}
Automotive radar imposes several constraints on antenna performance, guiding the interpretation of the state of the art:
\begin{itemize}
    \item \emph{Polarization purity:} Cross-polar discrimination (XPD) of $>20\ \mathrm{dB}$ across $\pm 60^\circ$ field-of-view (FoV).
    \item \emph{Integration:} Compatibility with compact radar front-ends, low-profile PCB or package-level integration.
    \item \emph{Mutual coupling:} Sufficient isolation between MIMO elements to enable virtual aperture synthesis.
    \item \emph{Manufacturability and cost:} Feasible process for high-volume production.
    \item \emph{Thermal stability:} Minimal gain/phase drift under automotive temperature cycles.
\end{itemize}

% ============================================================================
\section{Overview of radar front-end technology}
\label{sec:ant_overview}
\todo{Improve flow or remove this section\ldots}
\todo{Add sources\ldots}
This section introduces the principal antenna families explored in recent literature, organized by structural and electromagnetic characteristics relevant to polarimetric performance.

\textbf{Microstrip patch antennas} are extensively used in commercial 77~GHz radar due to their compatibility with multilayer PCB manufacturing. Dual-polarized versions are achieved using orthogonal feeds (probe, aperture, or proximity-coupled). Their main drawbacks are limited bandwidth and sensitivity to fabrication tolerances, which directly degrade polarization purity.

\textbf{Substrate integrated waveguide} (SIW) and \textbf{LTCC waveguide}-based antennas exhibit low loss and excellent XPD. Their robustness against temperature-induced deformations makes them suitable for high-frequency radars. However, their increased process complexity may be undesirable for large MIMO arrays unless the design is highly modular.

\textbf{Gap waveguide} (GW) and \textbf{ridge gap waveguide} (RGW) technology provides low loss, high isolation, and excellent dual-polarization characteristics. It avoids the need for electrical contact between layers, which eases millimetre-wave fabrication. This class currently represents one of the most promising platforms for dual-polarized MIMO.

\textbf{Slot-based antennas} (cavity-backed slots, CP-slots, and slanted slot arrays) offer naturally high polarization purity and robustness. Their main challenges relate to bandwidth and integration complexity when a wide field of view is required.

\textbf{Horn and horn-like antennas} exhibit outstanding gain and polarization purity but are impractical for system-level automotive integration. They are most relevant as reference antennas or measurement standards.

\textbf{Dielectric resonator and rod antennas} (DRAs) and rod antennas offer wide bandwidth and high radiation efficiency. The main challenges remain mechanical robustness and integration in thin radar package stacks.

% ============================================================================
\section{Comparative analysis of antenna technologies}
\label{sec:ant_table}
The following table summarizes representative antenna types found in the literature and highlights key parameters that influence their suitability for polarimetric automotive MIMO arrays.

\begin{landscape}
\begin{longtable}{p{4.8cm} p{1cm} p{1.6cm} p{1.6cm} p{1.6cm} p{1.6cm} p{1.8cm} p{1.8cm} p{5.4cm} }
\caption{\label{tab:antenna_comparison}Comparison of antenna types for 77--81~GHz polarimetric MIMO radar}\\
\toprule
\textbf{Antenna type} & \textbf{DP} & \textbf{BW} & \textbf{Loss} & \textbf{XPD} & \textbf{SLL} & \textbf{Isolation} & \textbf{Compact} & \textbf{Other considerations} \\
\midrule
\endfirsthead

\toprule
\textbf{Antenna type} & \textbf{DP} & \textbf{BW} & \textbf{Loss} & \textbf{XPD} & \textbf{SLL} & \textbf{Isolation} & \textbf{Compact} & \textbf{Other considerations} \\
\midrule
\endhead

\midrule
\multicolumn{9}{r}{Continued on next page}\\
\endfoot

\bottomrule
\endlastfoot

SIW-aperture-coupled microstrip patch antenna~\parencite{puskely5GSIWBasedPhased2022,animHighGainMillimeterWavePatch2021,karamiDevelopingBroadbandMicrostrip2022a} & Yes & Narrow & Low & High & Moderate & Moderate & Yes & PCB-compatible, tight spacing, thermal concerns \\
\hline
Series-fed patch antenna array~\parencite{baur77GHzAutomotive2013} & Yes & Narrow & High & Moderate & High & Moderate & Yes & Cost-effective, beam squint at wide scan \\
\hline
Series-parallel-fed square patches~\parencite{hambergerMixedCircularLinear2019} & Yes & Moderate & High & High & Moderate & Moderate & Yes & TMD-MIMO, single CP or dual LP capabilities \\
\hline
Series-fed aperture-coupled patches~\parencite{vallecchiDesignDualpolarizedSeriesfed2005} & Yes & Narrow & Low & Very high & Low & High & Yes & -- \\
\hline
Series-fed vertically-loaded multi-layer patch array~\parencite{aliakbari79GHzMultilayer2022} & Yes & Wide & Moderate & High & Moderate & High & Yes & Stacked vias reduce surface wave and widen BW \\
\hline
Corrugated sectoral horn antenna array~\parencite{trummerPerformanceAnalysis792017,tintiFullyPolarimetricAutomotive2024} & No & Wide & Low & Very high & Very low & Very high & No & Bulky, high cost, integration challenges \\
\hline
Ridge gap-waveguide polarimetric antennas~\parencite{zang77GHzFully2024,renAutomotivePolarimetricRadar2024} & Yes & Moderate & Low & Very high & Low & Very high & Yes & Excellent PCB integration, good FoV \\
\hline
Mixed edge- and aperture-fed multi-layer patch antenna~\parencite{acimovicDualPolarizedMicrostripPatch2008} & Yes & Narrow & High & High & Moderate & High & Yes & Complex feeding \\
\hline
L-shaped horn antenna array~\parencite{schulwitzMillimeterWaveDualPolarized2006} & Yes & Wide & Low & Very high & Moderate & High & No & Good PCB integration, lower gain \\
\hline
Cavity-backed ring-slot slot planar antenna~\parencite{ferreiraCompactDualCircularlyPolarized2017} & Yes & Narrow & Moderate & Very high & Moderate & Moderate & Yes & High F/B ratio, circular polarization \\
\hline
Cavity-slotted waveguide antenna~\parencite{sunWidebandCavitySlottedWaveguide2024} & No & Wide & Low & Very high & Low & High & Yes & -- \\
\hline
Dielectric rod antenna~\parencite{guoDualpolarizedDielectricResonator2003,kishkWidebandTruncatedTetrahedron2003} & Yes & Wide & Low & High & Moderate & Moderate & No & Robust design \\
\hline
Hybrid thin-film multi-layer antenna~\parencite{khanHybridThinFilm2017} & No & Wide & Very high & Low & High & Low & Yes & Poor efficiency ($<40\%$), limited beamwidth \\
\hline
LP comb-line microstrip stub array~\parencite{tintiFullPolarimetricAntenna2023} & No & Narrow & High & Limited & Moderate & Moderate & Yes & Decent XPD only near boresight \\
\hline
Slotted SIW array with bent slots~\parencite{bauer79GHzResonantLaminated2013} & No & Narrow & Low & Very high & Low & High & No & Minor beam squint, LTCC \\
\hline
Stepped SIW quasi-pyramid horn antenna~\parencite{zhaoSIWQuasiPyramidHorn2024} & Yes & Wide & Low & Very high & Low & High & Yes & High gain, emerging design \\
\hline
Cavity-backed bow-tie patches~\parencite{mosalanejadMultiLayerPCBBowTie2020} & Yes & Moderate & Moderate & High & Moderate & Moderate & Yes & Good FoV \\
\hline
Spline-shaped microstrip edge-fed antennas~\parencite{salucciSplineShapedMicrostripEdgeFed2024} & Yes & Narrow & Moderate & Moderate & Moderate & Moderate & Yes & Simple structure, easy fabrication \\
\bottomrule
\end{longtable}
\end{landscape}

% ............................................................................
\paragraph{Insights.} Based on the literature survey and the quantitative comparison in Table~\ref{tab:antenna_comparison}, several conclusions can be drawn for the design of a polarimetric MIMO array:
\begin{enumerate}
    \item \emph{Gap waveguide antennas provide the best balance of XPD, mutual coupling, and thermal stability}, making them strong candidates for dual-polarization MIMO demonstrators.
    \item \emph{Microstrip patches remain attractive for prototyping} due to ease of fabrication, but their limited polarization purity and sensitivity to tolerances make them challenging for high-fidelity polarimetry.
    \item \emph{SIW/LTCC designs offer excellent isolation} and are compatible with module-level integration, though process complexity must be managed.
    \item \emph{Slot antennas exhibit robust polarization purity}, but may require careful bandwidth enhancement for wide FoV radar.
\end{enumerate}

\begin{colourbox}[orange]{Additional notable technology platforms}
\begin{itemize}
    \item Metasurface/metamaterial antennas: Antennas leveraging metasurfaces can offer tailored polarization control and mutual coupling reduction. This technology is promising for miniaturization and XPD enhancement though requires further experimental validation in automotive radar contexts.
    \item Reconfigurable antennas: Utilizing tunable materials or MEMS can enable adaptive polarization states and beam steering, potentially benefiting polarimetric MIMO performance and multifunctionality.
    \item Low noise and active antennas: Integration with active circuitry (LNA or phase shifters) at the antenna element level improves noise figure and overall system sensitivity but complicates design and thermal management.
    \item 3D-printed antennas for millimetre-wave: Emerging additive manufacturing allows complex geometries with lightweight materials, which might open new design spaces for compact, high-performance antennas suitable for radar demonstrators.
\end{itemize}    
\end{colourbox}

% ............................................................................
\paragraph{Implications.} The findings guide the selection of antenna elements for the polarimetric MIMO demonstrator developed in this project. High-XPD, low-coupling technologies such as SIW and gap-waveguide antennas appear most capable of supporting robust polarimetric measurements and scalable sparse-MIMO topologies. Practical aspects such as manufacturability, integration with the transceiver front-end, and thermal robustness further influence the final choice.