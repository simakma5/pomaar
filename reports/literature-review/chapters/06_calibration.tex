\chapter{Calibration methods}
\label{chap:calibration}
Calibration of polarimetric radar systems is essential due to inevitable amplitude/phase imbalance, mutual coupling and cross-polar leakage. While well-established calibration routines exist for SAR and weather radar, few automotive-compatible solutions have been documented. The combination of TDM-MIMO sequencing, dual-polarized antennas and PCB-based front-end electronics creates significant channel non-idealities that classical methods do not address. The recent introduction of compact polarimetric reflectors and near-field calibration targets offers a starting point, but a consistent calibration methodology for $\qtyrange{77}{81}{GHz}$ polarimetric MIMO arrays remains an open research topic.

\section{MIMO calibration}

\section{Polarimetric calibration}

\section{Automotive constraints}

\begin{custombox}[orange]{Notes}
\begin{itemize}
    \item Not the main focus of my research -- mention potential sources
    \item Traditional array calibration: Vasanelli, Buitrago, and Harter (unread)
    \item Polarimetric: Changxu, Visentin, and Weishaupt
\end{itemize}
\end{custombox}
