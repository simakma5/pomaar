\chapter{Introduction}
Automotive radar is a rapidly-evolving field of research, driven by the increasing demand for enhanced driver safety and the eventual realization of autonomous vehicles~\parencite{waldschmidtAutomotiveRadarFirst2021}. While the initial vision of fully autonomous driving has faced regulatory and technological hurdles, the major focus of the industry has pragmatically shifted. The immediate goal is now the improvement of driver safety through advanced driver-assistance systems (ADAS). This forms a crucial foundational step, with the progressive automation of further features over time being the pathway towards full vehicle autonomy.

Modern ADAS increasingly rely on robust environmental perception to ensure safety and reliable operation across a wide range of road conditions. Among all sensing modalities, millimetre-wave radar at 77--81~GHz remains uniquely resilient to adverse weather, low visibility, and non-cooperative targets. While contemporary automotive radars provide accurate range, velocity and angular estimates, their classification capability remains fundamentally limited. Radar systems in vehicles are predominantly single-polarized MIMO arrays, optimized for geometric detection but not for detailed scattering characterization. As a result, the ability to distinguish vulnerable road users (VRUs) -- including pedestrians, cyclists, and scooter riders -- from background clutter or other vehicles is still insufficient for high-level scene understanding~\parencite{tillyRoadUserClassification2021}.

In contrast, radar polarimetry has long demonstrated powerful discrimination capabilities in other domains such as remote sensing, weather radar, and synthetic aperture radar (SAR). Polarimetric measurements capture how objects transform the polarization state of the incident electromagnetic field, thereby revealing structural, material and orientation-dependent scattering properties. These additional degrees of freedom can support classification tasks that are fundamentally unattainable with intensity-only radar data. Yet, despite its maturity in other fields, polarimetry has not been adopted in commercial automotive radar. Early research prototypes indicate its potential, but the fundamental theoretical, technological and calibration challenges remain unresolved.

Furthermore, the transition to high-resolution 2D MIMO arrays in automotive radars introduces new opportunities for capturing polarization-diverse scattering. Larger virtual apertures, wide angular fields of view and sparse array concepts can benefit polarimetric imaging -- but they also exacerbate issues such as cross-polar coupling, mutual coupling between antennas, phase-centre misalignment and channel imbalance. These limitations motivate a systematic reassessment of array topologies, antenna technologies, and processing schemes required for polarimetric MIMO radar at W-band.

This literature review therefore aims to synthesize the state of the art in:
\begin{itemize}
    \item radar polarimetry theory and its applicability to automotive environments,
    \item antenna technologies suitable for dual-polarization operation at 77--81~GHz,
    \item MIMO array topologies and multiplexing schemes relevant for polarimetric systems, and
    \item signal-processing and calibration methods required for polarimetric reconstruction.
\end{itemize}

The synthesis identifies critical gaps in current knowledge -- most notably the lack of a polarimetric method suitable for dynamic road scenes, Doppler-resolved polarimetry at mmWave, and MIMO array designs optimized jointly for polarization purity and automotive integration. These insights form the basis for the research contributions of this PhD: the development of a new polarimetric method for automotive radar, the design of a polarimetric MIMO array demonstrator, and the establishment of a dedicated measurement experiment for extracting target polarimetric signatures.
